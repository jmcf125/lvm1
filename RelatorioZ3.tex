\documentclass[a4paper,12pt]{article}
\begin{document}
\title{Projeto 1: SAT}
\author{João Miguel \and Rui Breda}
\date{\today}
\maketitle
\section{Sudoku}
\subsection{Classic Sudoku}
\paragraph{Descrição do problema \newline}
Este exercício do projeto consistiu na modelação e redução do problema do sudoku para SAT. Este problema consiste no seguinte: dado um tabuleiro inicial 9x9, preencher todas as células do tabuleiro com números de 1 a 9 tal que nenhum número se repita na mesma linha, coluna e região (quadrado 3x3).
\paragraph{Símbolos Proposicionais \newline}
Uma abordagem à resolução deste problema através do SAT-solver z3 implica que este seja descrito como fórmulas de símbolos proposicionais em CNF (Conjunctive Normal Form). Fórmulas estas que serão testadas face diversas atribuições aos símbolos proposicionais.\newline

Uma vez que se pretende explicitar que nenhuma célula pode ter um número já presente na mesma linha, coluna e região, o número de símbolos proposicionais não poderia ser inferior a 9$\times$9. Como o domínio onde trabalhamos é booleano, cada número que as células podem ter terá que ser um símbolo proposicional, com o valor verdadeiro se o número se encontra na célula e falso coso contrário.\newline
 
Portanto, haverá no total um símbolo por cada célula e por cada número possível, resultando em 9$\times$9$\times$9 símbolos. Assim, os símbolos terão a forma p\textunderscore i\textunderscore j\textunderscore n, 0 $\leq$ i $\leq$ 8 , 0 $\leq$ j $\leq$ 8 e 1 $\leq$ n $\leq$ 9 com i e j os índices da célula no tabuleiro e n um número possível de 1 a 9.
\paragraph{Condições e restrições \newline}
Estabelecidos os símbolos proposicionais, é necessário descrever o problema em termos de condições e restrições ás quais este será sujeito.
Como foi referido anteriormente, cada célula necessita de ter um e um só número. Esta condição pode ser dividida em duas:\newline
\begin{itemize}
	\item
	Cada célula tem pelo menos um número, ou seja, os símbolos da respetiva célula e dos diversos números não podem ser todos simultaneamente falsos
	
	\begin{displaymath}
	\
	\bigwedge_{0\leq i,j < 9} \bigvee_{1\leq n \leq 9} p \textunderscore i \textunderscore j \textunderscore n
	\end{displaymath}
	
	 \item Cada célula tem no máximo um número. A presença desta condição é justificada pelo facto de, se houver mais do que um número por célula, os restantes números da linha, coluna e região serão influenciados.
	
	\begin{displaymath}
	\
	\bigwedge_{0\leq i,j < 9} \bigwedge_{1\leq k < m < 9} \lnot p \textunderscore i \textunderscore j \textunderscore k \bigvee \lnot p \textunderscore i \textunderscore j \textunderscore m
	\end{displaymath}
	
	\item Feitas as condições para as células, é necessário implementar as restrições que ditam as regras do puzzle.
	As seguintes restrições  indicam que uma linha e uma coluna não podem ter um número repetido, com a única diferença entre as duas ser os índices de p.
	
	\begin{displaymath}
	\
	\bigwedge_{0\leq i < 9} \bigwedge_{1\leq k \leq 9} \bigwedge_{0\leq j < m < 9} \lnot p \textunderscore i \textunderscore j \textunderscore k \bigvee \lnot p \textunderscore i \textunderscore m \textunderscore k
	\end{displaymath}
	
	\begin{displaymath}
	\
	\bigwedge_{0\leq i,j < 9} \bigwedge_{1\leq k < m < 9} \lnot p \textunderscore i \textunderscore j \textunderscore k \bigvee \lnot p \textunderscore i \textunderscore j \textunderscore m
	\end{displaymath}
	
	
\end{itemize}

\subsection{}




\end{document}