\documentclass[a4paper,12pt]{beamer}

\usepackage[portuguese]{babel}
\usepackage{tikz}    % diagramas
\usepackage{amsmath} % align
\usepackage{amssymb} % mathbb, varnothing
\usepackage{amsthm}  % teoremas
\usepackage{xspace}
\usepackage{logicpuzzle}

\theoremstyle{definition}
\newtheorem{defn}{Definição}[section]
\newtheorem*{defn-alt}{Definição alternativa}
\newtheorem{ax}[defn]{Axioma}

\theoremstyle{theorem}
\newtheorem{teor}[defn]{Teorema}
\newtheorem{prop}[defn]{Proposição}
\newtheorem{conj}[defn]{Conjectura}
\newtheorem{lema}[defn]{Lema}
\newtheorem{coro}[defn]{Corolário}

\theoremstyle{remark}
\newtheorem{nota}[defn]{Nota}
\newtheorem{notç}[defn]{Notação}
\newtheorem{expl}[defn]{Exemplo}
\newtheorem{cexpl}[defn]{Contra-exemplo}

\usetikzlibrary{arrows.meta} % para mudar o tamanho das cabeças das setas

\newcommand{\seta}{\draw[-{>[scale=2,width=2]},line width=0.5pt]}
\newcommand{\sums}{\texttt{SUMS}\xspace}
\newcommand{\sat}{\texttt{SAT}\xspace}
\newcommand{\np}{\textbf{NP}\xspace}
\newcommand{\code}[1]{\texttt{#1}}

\newenvironment{sudoku}[1][]{%
\begin{logicpuzzle}[rows=9,columns=9,#1]
	\begin{puzzleforeground}
		\framepuzzle
		\framearea{black}{(1,1)--(4,1)--(4,4)--(1,4)--cycle}
		\framearea{black}{(4,1)--(7,1)--(7,4)--(4,4)--cycle}
		\framearea{black}{(7,1)--(10,1)--(10,4)--(7,4)--cycle}
		\framearea{black}{(1,4)--(4,4)--(4,7)--(1,7)--cycle}
		\framearea{black}{(4,4)--(7,4)--(7,7)--(4,7)--cycle}
		\framearea{black}{(7,4)--(10,4)--(10,7)--(7,7)--cycle}
		\framearea{black}{(1,7)--(4,7)--(4,10)--(1,10)--cycle}
		\framearea{black}{(4,7)--(7,7)--(7,10)--(4,10)--cycle}
		\framearea{black}{(7,7)--(10,7)--(10,10)--(7,10)--cycle}
	
	\end{puzzleforeground}
}{\end{logicpuzzle}}

\title{Projeto 1: \sat em {Z3}}
\author{João Miguel Faria \and Rui Breda Perdigoto}
\date{\today}

\begin{document}

\begin{frame}
   \titlepage
\end{frame}

\begin{frame}
   \frametitle{Índice}
   \tableofcontents%[pausesections]
\end{frame}

\section{Sudoku}

\subsection{Símbolos Proposicionais}
\begin{frame}
\frametitle{Símbolos Proposicionais}
Símbolos proposicionais 
   \begin{align*}
      &p_{i,j,n}\\
      0&\leq i, j \leq 8\\
      1&\leq n \leq 9
   \end{align*}
\end{frame}

\subsection{Condições e restrições}
\begin{frame}
\frametitle{Condições e restrições}
\begin{itemize}
     \item 1 número por célula
        \begin{align}
        \bigwedge_{0\leq i,j < 9}
           \bigvee_{1\leq n \leq 9} p_{i,j,n}
        \end{align}
     
        \begin{align}
        \bigwedge_{0\leq i,j < 9}
           \bigwedge_{1\leq k < m < 9}
           {\lnot p_{i,j,k} \lor \lnot p_{i,j,m}}
        \end{align}
        \pause
      \item Linhas e colunas sem repetições
         \begin{align}
         \bigwedge_{0\leq i < 9} \bigwedge_{1\leq k \leq 9} \bigwedge_{0\leq j < m
         < 9} \lnot p_{i,j,k}
         \lor \lnot p_{i,m,k}
         \end{align}

         \begin{align}
         \bigwedge_{0\leq i < 9} \bigwedge_{1\leq k < 9} 
            \bigwedge_{0\leq j < m < 9}
               \lnot p_{j,i,k} \land \lnot p_{m,i,k}
         \end{align}
         
\end{itemize}
\end{frame}

\begin{frame}
\begin{itemize}
     \item Regiões sem repetições
        %$c$ é o índice da primeira coluna da
        %região e $b = i+(2 - i \bmod 3)$, que representa o índice da última
        %linha da região.
     \begin{align}
        \bigwedge_{0\leq i < 9}
        \bigwedge_{1\leq k \leq 9}
        \bigwedge_{0\leq j < 9}
        \bigwedge_{i + 1 \leq n < b}
        \bigwedge_{c\leq l < c+3} \lnot p_{i,j,k} \lor \lnot p_{n,l,k}
     \end{align}
     
        \pause
     \item $S$ com os símbolos do tabuleiro inicial
        \begin{align}
        \bigwedge_{p\in S} p
        \end{align}
     
        \pause
     \item \texttt{well\_posed(P)} tem uma
        restrição extra, dado um conjunto $M$ com uma solução
        \begin{align}
        \bigvee_{p\in M} \lnot p
        \end{align}
\end{itemize}
\end{frame}

\subsection{Execução do programa}

\begin{frame}
\frametitle{Execução do programa}
\begin{figure}
	\caption{Tabuleiro inicial}
	\centering
	\begin{sudoku}[scale=.5]
		\setrow{1}{, , , , 8, , , 7, 9}
		\setrow{2}{, , , 4, 1, 9, , , 5}
		\setrow{3}{, 6, , , , , 2, 8, }
		\setrow{4}{7, , , , 2, , , , 6}
		\setrow{5}{4, , , 8, , 3, , , 1}
		\setrow{6}{8, , , , 6, , , , 3}
		\setrow{7}{, 9, 8, , , , , 6, }
		\setrow{8}{6, , , 1, 9, 5, , , }
		\setrow{9}{5, 3, , , 7, , , , }
	\end{sudoku}
\end{figure}
\end{frame}

\begin{frame}
\begin{figure}
   \caption{Puzzle resolvido após \code{sudoku(P)}}
	\centering
	\begin{sudoku}[scale=.5]
	\setrow{1}{3, 4, 5, 2, 8, 6, 1, 7, 9}
	\setrow{2}{2, 8, 7, 4, 1, 9, 6, 3, 5}
	\setrow{3}{9, 6, 1, 5, 3, 7, 2, 8, 4}
	\setrow{4}{7, 1, 3, 9, 2, 4, 8, 5, 6}
	\setrow{5}{4, 2, 6, 8, 5, 3, 7, 9, 1}
	\setrow{6}{8, 5, 9, 7, 6, 1, 4, 2, 3}
	\setrow{7}{1, 9, 8, 3, 4, 2, 5, 6, 7}
	\setrow{8}{6, 7, 2, 1, 9, 5, 3, 4, 8}
	\setrow{9}{5, 3, 4, 6, 7, 8, 9, 1, 2}
	\end{sudoku}
\end{figure}
\end{frame}

%Ao correr a função \code{well\_posed} com o tabuleiro inicial do exemplo
%anterior leva a \code{unsat}, o que indica que este é \code{well\_posed}, ou
%seja, possui
%apenas uma solução. Correndo \code{well\_posed} com um tabuleiro inicial mais
%benevolente como o que se apresenta leva a que este possua múltiplas soluções,
%pelo que não é \code{well\_posed}.


\begin{frame}
\begin{figure}
   \caption{Tabuleiro inicial de \code{well\_posed}}
	\centering
	\begin{sudoku}[scale=.5]
		\setrow{1}{,,,,,,,,}
		\setrow{2}{,9,,,,8,,,}
		\setrow{3}{,,,,,,,6,}
		\setrow{4}{,,4,3,,,,,}
		\setrow{5}{,,,,,,,,}
		\setrow{6}{,,,,2,,,,}
		\setrow{7}{,,,,,9,,,}
		\setrow{8}{,,2,,,,,,}
		\setrow{9}{1,,,,,,7,,}
	\end{sudoku}
\end{figure}
\end{frame}


\subsection{Variantes}
\begin{frame}
\frametitle{Variantes}
Células vizinhas \---- \code{deltas\_dist}

\code{deltas\_vals=[0]} \---- células vizinhas $\neq$ célula atual
\end{frame}

\begin{frame}
\begin{notç}
   \begin{align}
      i,j,n &\in\mathbb N\cap[1,9]\\
      (c_{i,j}=k) &:= p_{i,j,k}
   \end{align}
   não induz em erro porque há restrições usuais, temos por exemplo:
   \begin{align}
      c_{i,j}=k\land c_{i,j}=k' \implies k = k'
   \end{align}
\end{notç}
\end{frame}

\begin{frame}
\begin{defn}
\begin{align}
   \mathtt{diferentes}(\Delta, V) :=%\nonumber\\
   &\bigwedge_{i,j,n}
   \bigwedge_{\substack{c'\in\Delta+'c_{i,j}\\\nu\in V}}
   (c_{i,j} = n
      \implies
   c' \neq n + \nu)
\end{align}
\pause
onde
\begin{align}
   \delta &= (\delta_1,\delta_2)\in \Delta\\
   \Delta+'c &= \{c_{(i,j)+\delta}:\delta\in\Delta\}\cap\{1,\cdots,9\}^2
\end{align}
para garantir que não saímos do tabuleiro.
\end{defn}
\end{frame}

\section{\sums}
\subsection{Descrição do problema}
\begin{frame}
\frametitle{Descrição do problema}
Dados:
\begin{align*}
   R&\subseteq\mathbb N\\
   t &\in\mathbb N
\end{align*}
\pause

Queremos $\exists S\subseteq R$ tal que $\sum S = t$.
\end{frame}

\subsection{Observações sobre a Complexidade}
\begin{frame}
\frametitle{Observações sobre a Complexidade}
\begin{itemize}
   \item $p_r$ para cada $r\in R$
      \---- exponencial ao criar restrições (\code{sums\_red\_exp})
      \pause
   \item $p_S$ para cada $S\subseteq R$
      \---- exponencial no número de proposições
\end{itemize}
      \pause

Queremos uma redução polinomial!
\end{frame}

\subsection{Símbolos proposicionais e restrições}
\begin{frame}
\frametitle{Símbolos proposicionais e restrições}
\begin{notç}
   Nesta secção tomamos sempre
   \begin{align}
      u,w&\in R\\
      r,s&\in \mathbb N\cap[0,t]
   \end{align}
\end{notç}
\end{frame}

\begin{frame}
Ideia:
   \begin{align}
   p_{u,r}
   \end{align}
$u$ \---- último dígito lido

$r$ \---- resto até $t$

$\#R\times(t+1)$ símbolos proposicionais

\pause
\begin{align}
   \bigwedge_{r\neq0}
   \bigwedge_{u}\left( p_{u,r}
   \implies
   \bigvee_{w}p_{w,r-w}
   \right)
\end{align}
Podemos ler $w$ e ficar $w$ unidades mais perto de $t$.
\footnote{
   $r\neq 0$ porque $\bigvee\varnothing=\bot$,
   não queremos $p_{u,0}\implies\bot \forall u$!
}
\end{frame}


\begin{frame}
\begin{figure}
%\begin{minipage}{0.4\textwidth}
   \centering
   \begin{tikzpicture}[node distance=2cm]
      % níveis 0 e 1
      \node(p43)                    {$\boldsymbol{p_{4,3}}$};
      \node(p25)  [left of=p43]     {$\boldsymbol{p_{2,5}}$};
      \node(p16)  [left of=p25]     {$\boldsymbol{p_{1,6}}$};
      \node(p07)  [above of=p25] {$(p_{\varnothing,7})$};

      \seta(p07) -- (p16);
      \seta(p07) -- (p25);
      \seta(p07) -- (p43);

      % nível 2   -- criar relevo aqui
      \node(x1)  [below left of=p16]   {$\vdots$};
      \node(x2)  [below right of=x1]   {$\vdots$};
      \node(x3)  [right of=x1]         {$\vdots$};
      \node(x4)  [below right of=x3]   {$\vdots$};
      \node(p12) [below left of=p43]   {$p_{1,2}$};
      \node(p21) [below right of=p43]  {$p_{2,1}$};

      \seta(p16) -- (x1);
      \seta(p16) -- (x2);
      \seta(p25) -- (x3);
      \seta(p25) -- (x4);
      \seta(p43) -- (p12);
      \seta(p43) -- (p21);

      % nível 3
      \node(p10) [below right of=p12]  {$p_{1,0}$};
      \node(p20) [below right of=p21]  {$p_{2,0}$};

      \seta(p12) -- (p10);
      \seta(p21) -- (p20);
   \end{tikzpicture}
   \caption{Exemplo de uma árvore a partir de $R=\{1,2,4\}$ e
      $t=7$.}
%\end{minipage}
%\hfill
\end{figure}
\end{frame}

\begin{frame}
Não temos $p_{\varnothing,r}$; começamos em:
\begin{align}
   \bigvee_{u} p_{u,t}
\end{align}

\pause
Queremos chegar a:
\begin{align}
   \bigvee_{u} p_{u,0}
\end{align}
\end{frame}

\begin{frame}
Não é preciso $\bigwedge_{u\neq w} p_{u,0}\to\lnot p_{w,0}$, temos:
\begin{align}
      \bigwedge_{u,r}\left( p_{u,r}
   \implies
      \bigwedge_{w\neq u}\lnot p_{w,r}
         \land
      \bigwedge_{s\neq r}\lnot p_{u,s}
   \right)
\end{align}
que já deita extras (expl. permutações, vários fins) fora.
\end{frame}

\begin{frame}
Podemos dizer (mas não é preciso):
\begin{align}
   \bigwedge_{u+r>t} \lnot p_{u,r}
\end{align}
\end{frame}

\subsection{Um exemplo}
\begin{frame}[fragile] %https://tex.stackexchange.com/questions/140719/verbatim-in-beamer-showing-error-file-ended-while-scanning-use-of-xverbatim
\frametitle{Um exemplo}
Para o caso
\begin{align*}
   R&=\{0,1,2,3,4,5,8,16,32,64,500\}\\
   t&=74
\end{align*}
\texttt{sums(R,t)} resulta
em:
\begin{verbatim}
[p_2_0, p_4_70, p_3_67, p_64_2, p_1_66]
[2, 4, 3, 64, 1]
\end{verbatim}
\end{frame}
\end{document}
